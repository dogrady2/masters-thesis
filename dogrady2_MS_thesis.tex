%%
%% This is file `thesis-ex.tex',
%% generated with the docstrip utility.
%%
%% The original source files were:
%%
%% uiucthesis2009.dtx  (with options: `example')
%% 
\def\fileversion{v1.0} \def\filedate{2018/07/20}
%% Package and Class "uiucthesis2009" for use with LaTeX2e.
\documentclass[edeposit,fullpage]{uiucthesis2009}
\usepackage[acronym,toc]{glossaries}
%\newacronym{<++>}{<++>}{<++>}
%\newacronym{<++>}{<++>}{<++>}

\newacronym{MTHM}{MTHM}{metric ton of heavy metal}
\newacronym{XS}{XS}{cross-sections}
\newacronym{AO}{AO}{Axial Offset}
\newacronym{PCI}{PCI}{Pellet-Cladding Interaction}
\newacronym{PCMI}{PCMI}{Pellet-Cladding Material Interaction}
\newacronym{lhr}{LHR}{Linear Hear Rate}
\newacronym{SCC}{SCC}{Stress Corrosion Cracking}
\newacronym{MPS}{MPS}{Missing Pellet Surface}

\newacronym{BOC}{BOC}{Beginning of Cycle}
\newacronym{BOL}{BOL}{Beginning of Life}
\newacronym{HZP}{HZP}{Hot Zero Power}
\newacronym{HFP}{HFP}{Hot Full Power}
\newacronym{CASL}{CASL}{Consortium for Advanced Simulation of Light Water Reactors}
\newacronym{IFBA}{IFBA}{Integral Fuel Burnable Absorber}
\newacronym{UMich}{UMich}{University of Michigan}
\newacronym{NCSU}{NCSU}{North Carolina State University}
\newacronym{ORNL}{ORNL}{Oak Ridge National Laboratory}
\newacronym{CMFD}{CMFD}{Coarse Mesh Finite Diference}
\newacronym{INL}{INL}{Idaho National Laboratory}
\newacronym{MOOSE}{MOOSE}{Multiphysics Object Oriented Simulation Environment }
\newacronym{ITC}{ITC}{Isothermal Temperature Coefficient }
\newacronym{CRW}{CRW}{Control Rod Worth}




\newacronym{DOE}{DOE}{Department of Energy}
\newacronym{EPA}{EPA}{Environmental Protection Agency}
\newacronym{EFPD}{EFPD}{Effective Full Power Days}
\newacronym{EOC}{EOC}{End of Cycle}
\newacronym{GWDMT}{GWDMT}{GigaWatt Day per Metric Ton}
\newacronym{DRWM}{DRWM}{Dynamic Rod Worth Measurement}


\newacronym{LWR}{LWR}{Light Water Reactor}
\newacronym{LPPT}{LPPT}{Low Power Physics Tests}

\newacronym{MIT}{MIT}{the Massachusetts Institute of Technology}


\newacronym{NEUP}{NEUP}{Nuclear Energy University Programs}
\newacronym{NPRE}{NPRE}{Department of Nuclear, Plasma, and Radiological Engineering}
\newacronym{NRC}{NRC}{Nuclear Regulatory Commission}

\newacronym{PARCS}{PARCS}{Purdue Advanced Reactor Core Simulator}
\newacronym{PATHS}{PATHS}{PARCS Advanced Thermal Hydraulic Solver}
\newacronym{PWR}{PWR}{Pressurized Water Reactor}
\newacronym{BWR}{BWR}{Boiling Water Reactor}

\newacronym{UOX}{UOX}{uranium oxide}
\newacronym{UQ}{UQ}{uncertainty quantification}
\newacronym{US}{U.S.}{United States}
\newacronym{UIUC}{UIUC}{University of Illinois at Urbana-Champaign}
\newacronym{VV}{V\&V}{verification and validation}
\newacronym{VERA}{VERA}{Virtual Enviromnment for Reactor Analysis}
\newacronym{VERACS}{VERA-CS}{VERA Core Simulator}
\newacronym{BEAVRS}{BEAVRS}{Benchmark for Evaluation And Validation of Reactor Simulations}
\newacronym{BW}{B\&W-1484}{Babcock and Wilcox critical experiments Series 1484}



\begin{document}

\title{Investigation of Pellet Clad Interaction during Load-Follow\\
       Operation in a Pressurized Water Reactor using VERA-CS}
\author{Daniel John O'Grady}
\department{Nuclear, Plasma, and Radiological Engineering}
\schools{B.S., University of Illinois at Urbana-Champaign, 2017}
\msthesis
\advisor{Tomasz Kozlowski}
\degreeyear{2018}
\committee{Professor Tomasz Kozloski, Advisor\\Professor Katherine Huff}
\maketitle

\frontmatter

%% Create an abstract that can also be used for the ProQuest abstract.
%% Note that ProQuest truncates their abstracts at 350 words.
\begin{abstract}
This is a comprehensive study of caffeine consumption by graduate
students at the University of Illinois who are in the very final
stages of completing their doctoral degrees. A study group of six
hundred doctoral students\ldots.
\end{abstract}

%% Create a dedication in italics with no heading, centered vertically
%% on the page.
\begin{dedication}
To Father and Mother.
\end{dedication}

%% Create an Acknowledgements page, many departments require you to
%% include funding support in this.
\chapter*{Acknowledgments}

This project would not have been possible without the support of
many people. Many thanks to my adviser, Lawrence T. Strongarm, who
read my numerous revisions and helped make some sense of the
confusion. Also thanks to my committee members, Reginald Bottoms,
Karin Vegas, and Cindy Willy, who offered guidance and support.
Thanks to the University of Illinois Graduate College for awarding
me a Dissertation Completion Fellowship, providing me with the
financial means to complete this project. And finally, thanks to
my husband, parents, and numerous friends who endured this long
process with me, always offering support and love.

%% The thesis format requires the Table of Contents to come
%% before any other major sections, all of these sections after
%% the Table of Contents must be listed therein (i.e., use \chapter,
%% not \chapter*).  Common sections to have between the Table of
%% Contents and the main text are:
%%
%% List of Tables
%% List of Figures
%% List Symbols and/or Abbreviations
%% etc.

\tableofcontents
\listoftables
\listoffigures

%% Create a List of Abbreviations. The left column
%% is 1 inch wide and left-justified
\chapter{List of Abbreviations}

\begin{symbollist*}
\item[CA] Caffeine Addict.
\item[CD] Coffee Drinker.
\end{symbollist*}

%% Create a List of Symbols. The left column
%% is 0.7 inch wide and centered
\chapter{List of Symbols}

\begin{symbollist}[0.7in]
\item[$\tau$] Time taken to drink one cup of coffee.
\item[$\mu$g] Micrograms (of caffeine, generally).
\end{symbollist}

\mainmatter
\chapter{Introduction}

In the \gls{US}, nuclear power generation has high fixed costs and low variable costs. 
As a result, utilities have traditionally sought to operate nuclear stations at full power from \gls{BOC} to \gls{EOC}. 
More recently, the deregulation of the energy market and the emergence of intermittent renewable energy sources have caused load-follow operation to become a more attractive option for nuclear generation. 

The deregulation of the energy market has forced utilities to compete against each other to sell electricity within a region.
The beneficiary of this competition are the customers, who are guaranteed fair prices for electricity and will not be footing the bill for an inefficient/uneconomical utility project. %need citation
Nuclear stations, with low variable costs and high fixed costs, have typically been able to economically compete within deregulated markets because the typical plant lifetime of at least 20 years allows owners to spread out the fixed costs.
Recently, the low price of natural gas and government subsidized renewable energy sources, such as wind and solar, have made nuclear stations appear inefficient.   

%
%%%%%%%%%%%%%%%%%%%%%%%%%%%%%%%%%%%%%%%%%%%%%%%%%%%%%%%%%%%%%%%%%%%%%%%%%%%%%%%%%
\section{Background and Motivation}

In 2016, the \gls{US} had approximately 7\% of its total electricity generation coming from wind and solar power \cite{u.s_energy_information_administration_electricity_2016}. 
This share is likely to increase as the \gls{US} continues to move away from fossil fuels and towards a "greener" energy future.
As a result of this increase, in combination with the deregulation of the energy market, the price of electricity has become volatile. 
At certain times, the price of selling electricity within a region can even become negative, due a sudden increase in renewable energy output and a low market demand \cite{paraschiv_impact_2014}. 
In some areas, negative electricity prices are further increased due to the fact that large generating facilities would rather sell at a loss to avoid decreasing their power level. 
This preference is caused by the high capital cost and relatively low variable costs of large generating facilities \cite{lokhov_load-following_2011}.

Nuclear stations are typically one of these large generating facilities.
As a result of the large construction costs and the fixed number of staff members that must be on site at all times, most utilities prefer to keep a reactor at full power, as it is easiest to maintain constant power. 
If instead of remaining at full power, a nuclear station operated in load-following mode, could this increase the efficiency of the plant?
During load-follow operation, a nuclear station will vary its power output in response to the anticipated demand to better suit the market needs, stabilizing the price of electricity.
Theoretically, the current operating plants were all designed with the maneuverability to respond to such change in demand \cite{lokhov_technical_2011}.
In fact, many of the reactors in France already participate in load-following maneuvers with the help of grey control rods \cite{lokhov_technical_2011}.
Grey control rods are similar to standard \gls{PWR} control rods but have significantly less rod worth \cite{yousefpour_improvement_2000}.
The low rod worth allows them to be used for reactivity control without putting significant stress on the surrounding fuel. 
In the \gls{US}, grey rods are not present in \gls{PWR}, increasing the complexity of load-follow operation \cite{lokhov_technical_2011}.
  
To participate in load-follow operation in the \gls{US}, a \gls{PWR} can use the critical boron concentration to modify the power level while making minor control rod insertion to manage the core \gls{AO} \cite{lokhov_technical_2011}.  
This practice can lead to significant changes in local power levels throughout the core.
Significant changes in local power can cause fuel to swell or contract, due to thermal expansion \cite{gartner_power_1984}. %need citation
If a utility chooses to ramp down the reactor during times of low demand, or high supply, the fuel pellets will contract allowing the cladding to slowly creep down.
When the decision is made to return the reactor to a higher power level, the rate at which the power can be increased is limited due to the thermal expansion of the fuel pellet \cite{gartner_power_1984}. %need citation
If enough time has passed at low power and the cladding has crept down on the fuel pellet, a sudden expansion of the fuel pellet has the potential to cause fuel failure.
As a result, studies must be performed on the maximum power ramp rate to determine if load-follow operation introduces a significant risk to all fuel rods.

In this study, the market conditions under which it is economic for a \gls{PWR} to perform load-follow power maneuvers were investigated. 
As a baseload energy source, current market conditions are causing the premature shutdown of some nuclear power plants \cite{_exelon_????}. 
If load-follow power maneuvers are possible for these plants, they can become more competitive in their local market, possibly allowing them to remain in operation without government intervention.
Although, improving the economics of a nuclear reactor is a strong driving force for this study, it is also important to improve the compatibility of nuclear power with renewable energy sources. 
In the absence of energy storage, a complete shift away from fossil fuels will require renewable energy sources, and hopefully nuclear power, to provide all of the energy on the grid.
If it can be proven that traditional \gls{PWR}s can safely and economically respond to the changes in renewable energy output, the shift away from fossil fuel can be expedited. 
In addition, nuclear power working together with renewable energy sources can positively impact the public's opinion on nuclear power.

\section{Literature Review}


\chapter{PWR1}

\chapter{VERA-CS}

\chapter{Results and Discussion}

\chapter{Conclusions}

We conclude that graduate students like coffee.

\appendix*

\include{Appendix.tex}

\backmatter

\bibliography{thesisbib}

\chapter{Vita}

Juan Valdez was born\ldots.

\end{document}
\endinput
%%
%% End of file `thesis-ex.tex'.
